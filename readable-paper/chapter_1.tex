%   Filename    : chapter_1.tex 
\chapter{Introduction}
\label{sec:researchdesc}    %labels help you reference sections of your document

\section{Overview of the Current State of Technology}
\label{sec:overview}

The ability to read is a fundamental skill that is necessary for success in many areas of life. Unfortunately, there are many adults in the Philippines who have never learned to read or who have difficulty reading due to various reasons such as illiteracy, limited education, or learning disabilities. These individuals often face significant barriers to employment, education, and social participation, leading to a cycle of poverty and marginalization.

In 2019, the Philippines achieved a literacy rate of 96.5 \% for the segment of the population aged 10 and over according to the PSA’s Functional Literacy, Education and Mass Media Survey (FLEMMS), as reported in an article from Business World entitled, “Literacy rate estimated at 93.8\% among 5 year olds or older — PSA..” Literacy was defined as the ability to read and write “with understanding of simple messages in any language or dialect.” However, the same article notes that this was the same rate observed in 2013, a matter described as alarming by University of Asia and the Pacific Senior Economist Cid L. Terosa, stating that even minimal improvements should be expected especially after six years.

More recently, according to a report published by UNICEF in collaboration with UNESCO and the World Bank, the percentage of 10-year-olds in low- and middle-income countries who are unable to read is as high as 70\%. This figure has likely been affected by school closures brought about by the COVID-19 pandemic. The same report also stated that only 10\% of children in the Philippines were able to read simple text as of March 2022. Alarmingly, a separate report published by the World Bank in 2021 found that the rate of learning poverty - defined as the inability to read simple text by age 10 - in the Philippines was at 90\%. These statistics highlight the urgent need to address the education crisis in the Philippines and the need to further augment the country’s current literacy situation.

To address this problem, we propose the development of an automatic reading miscue detection system called Readable, specifically targeting the Hiligaynon language. Hiligaynon, also known as Ilonggo, is an Austronesian language spoken in the Western Visayas region of the Philippines, particularly in the provinces of Iloilo, Guimaras, Negros Occidental, and Capiz. It is one of the major languages of the Philippines, spoken by millions of people as a first or second language.
Our reading miscue detection system will utilize machine learning techniques including automatic speech recognition (ASR). ASR is a technology that allows computers to automatically recognize and transcribe spoken language, and it has made significant advances in recent years. However, it can still be challenging to achieve high levels of accuracy for some languages and accents, especially those that are underrepresented in ASR training data. By targeting local languages like Hiligaynon and designing our ASR system to work well for these languages, we can help ensure that our reading tutor is accessible and effective for non-reading adults in the Philippines.

While there are some similar applications like Google Read Along available for reading instruction, they may not be accessible or relevant for many non-reading adults in the Philippines due to language barriers or lack of internet connectivity. By targeting local languages like Hiligaynon and utilizing the benefits of natural language processing and machine learning techniques, our automatic reading tutor can provide personalized and effective reading instruction that is accessible and relevant for non-reading adults in the Philippines. By providing accessible, effective, and scalable reading education in Hiligaynon, we hope to improve the lives and prospects of non-reading adults in the Philippines and break the cycle of poverty and illiteracy. Children with strong literacy skills grow more consistently and confidently in their studies, and reading literacy is a crucial gateway to other learning areas such as the humanities, mathematics, and the sciences. By addressing learning poverty and promoting reading literacy, we can help ensure that children in the Philippines have the opportunity to reach their full potential and succeed in their studies.

\section{Problem Statement}
Given the current educational crisis our country is facing (\cite{areweoneducationrecovery?_2022}) and with the aim to further improve the current state of  literacy rate of our country (\cite{neil_2020}), the development of automatic reading tutor systems which entails building reading miscue detection systems and other related programs becomes relevant. Furthermore, the limited resources available for Hiligaynon in the context of speech processing technologies opens a good opportunity to attempt to make a contribution for the said domain of interest.

\section{Research Objectives}
\label{sec:researchobjectives}

\subsection{General Objective}
\label{sec:generalobjective}

The aim of this project is to develop a reading miscue detection system that would determine the acceptability of an input speech relative to a reference speech pattern.

\subsection{Specific Objectives}
\label{sec:specificobjectives}

Specifically, the project targets to:
\begin{enumerate}
   \item Train and model a DNN-based acoustic model for Hiligaynon.
   \item Use the developed acoustic model to derive phonemic transcriptions of the input speeches/audio.
   \item Determine the acceptability of a user's speech input, given a set of predetermined words to read, in terms of its deviation from reference transcriptions via forced alignment; judging is based on a set threshold score.
\end{enumerate}

\section{Scope and Limitations of the Research}
\label{sec:scopelimitations}

The system is specific to the Hiligaynon language. The words used in the audio data are limited to 2-3 syllable Hiligaynon words from the book “Hiligaynon Lessons” by Cecille L. Motus. Deviations are only measured word-by-word. In terms of the toolkit, the system is limited by the features offered by Kaldi - an open source speech recognition toolkit for speech recognition and signal processing.

\section{Significance of the Research}
\label{sec:significance}

This project aims to take one step towards developing a solution for improving the reading skill of Filipinos, particularly non reading adults. The authors also aim to contribute to the growing efforts of including Philippine local languages, specifically, Hiligaynon in literatures related to speech processing, particularly those relevant to the development of automated reading tutors.

